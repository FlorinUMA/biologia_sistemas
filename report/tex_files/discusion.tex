\section{Discusión}

En el transcurso de esta investigación sobre la megalencefalia, es crucial reconocer y comprender las limitaciones que acompañan a nuestro enfoque de estudio. Una limitación significativa de este proyecto reside en la falta de sujetos humanos reales para estudiar la megalencefalia. Dada la naturaleza de nuestro enfoque, nos basamos en datos y simulaciones, sin contar con la participación directa de individuos afectados. Aunque se ha empleado un enfoque riguroso utilizando modelos y datos disponibles, la ausencia de casos clínicos reales podría afectar la generalización de los resultados a situaciones específicas de pacientes con megalencefalia. Además, hay que tener en cuenta que no podemos asegurar que las comunidades sean correctas, ya que los valores de modularidad que se han obtenido no soy muy altos. Se recomienda la validación adicional mediante estudios clínicos con participantes reales en futuras investigaciones. 


% What's next: queda estudiar cómo afectan los términos descubiertos en el mundo médico para un posible tratamiento del fenotipo.

%% What's next

%%%%%(https://onlinelibrary.wiley.com/doi/full/10.1002/ajmg.a.33765?casa_token=3fiBDRsX3MMAAAAA%3AB_xuv6vRyRmP9SGcEqjkwGfwC7IjFX6Zl2LIPcrf4OXvIf-UKZiZPnVEgEEaMnpIx6bc_VIv8AlIuRdF)%%%%% FGF8

En contraste con la literatura de los genes que hemos encontrado, podemos observar que el gen FGF8 está estrechamente relacionado con varias comunidades, puesto que es importante en el desarrollo del cerebro y la cara. Se ha demostrado que tiene funciones dosis-dependientes en la regulación de los centros de patrón telencefálico y en el desarrollo craneofacial \cite{fgf8}. Otro gen también que se ha considerado importante es el gen SHH, ya que parece estar altamente conectado en la comunidad 2, y además se conoce que está involucrado en la formación temprana de patrones en el sistema nervioso en desarrollo \cite{winden_megalencephaly_2015}. Por último, se han identificado un tercer gen (AKT3) con mucha conectividad en la comunidad 1, y se conoce que diferentes mutaciones en AKT3 se asocian con diferentes fenotipos, entre ellos se encuentra la megalencefalia \cite{akt3}. Destacamos entre las mutaciones de este gen los siguientes síntomas: discapacidad intelectual, epilepsia, autismo, malformaciones capilares en la piel, entre otros.

Como hemos observado anteriormente, existe una alta correlación en los genes mostrados en nuestro trabajo con el fenotipo, e invitamos a futuros investigadores que sigan nuestra línea de trabajo y comprueben experimentalmente que los demás genes mostrados están relacionados con el fenotipo en cuestión, e idealmente, desarrollar un tratamiento.