\section{Discusión}


% Context
En este estudio se ha investigado el fenotipo de la Megalencefalia, una patología cerebral, y su relación con los síntomas. Se han utilizado herramientas avanzadas de análisis de redes para comprender mejor las conexiones y patrones entre la enfermedad y posibles genes relacionados. Los hallazgos del estudio incluyen la identificación de cinco comunidades marcadas y la elección del algoritmo de Louvain como el mejor para la división de grupos. Además, se encontró que la función \textit{Negative regulation of intracellular signal transduction} se encuentra en varias comunidades.

En general, los hallazgos del estudio apoyan el paradigma actual de que la megalencefalia está relacionada con una variedad de genes y que estas conexiones pueden ser exploradas y comprendidas a través del análisis de redes.

En el transcurso de esta investigación sobre la megalencefalia, es crucial reconocer y comprender las limitaciones que acompañan a nuestro enfoque de estudio. Una limitación significativa de este proyecto reside en la falta de sujetos humanos reales para estudiar la megalencefalia. Dada la naturaleza de nuestro enfoque, nos basamos en datos y simulaciones, sin contar con la participación directa de individuos afectados. Aunque se ha empleado un enfoque riguroso utilizando modelos y datos disponibles, la ausencia de casos clínicos reales podría afectar la generalización de los resultados a situaciones específicas de pacientes con megalencefalia. Además, hay que tener en cuenta que no podemos asegurar que las comunidades sean correctas, ya que los valores de modularidad que se han obtenido no soy muy altos. Se recomienda la validación adicional mediante estudios clínicos con participantes reales en futuras investigaciones. 


% What's next: queda estudiar cómo afectan los términos descubiertos en el mundo médico para un posible tratamiento del fenotipo.

%% What's next

%%%%%(https://onlinelibrary.wiley.com/doi/full/10.1002/ajmg.a.33765?casa_token=3fiBDRsX3MMAAAAA%3AB_xuv6vRyRmP9SGcEqjkwGfwC7IjFX6Zl2LIPcrf4OXvIf-UKZiZPnVEgEEaMnpIx6bc_VIv8AlIuRdF)%%%%% FGF8

En contraste con la literatura de los genes que hemos encontrado, podemos observar que el gen FGF8 está estrechamente relacionado con varias comunidades, puesto que es importante en el desarrollo del cerebro y la cara. Se ha demostrado que tiene funciones dosis-dependientes en la regulación de los centros de patrón telencefálico y en el desarrollo craneofacial \cite{fgf8}. Otro gen también que se ha considerado importante es el gen SHH, ya que parece estar altamente conectado en la comunidad 2, y además se conoce que está involucrado en la formación temprana de patrones en el sistema nervioso en desarrollo \cite{winden_megalencephaly_2015}. Por último, se han identificado un tercer gen (AKT3) con mucha conectividad en la comunidad 1, y se conoce que diferentes mutaciones en AKT3 se asocian con diferentes fenotipos, entre ellos se encuentra la megalencefalia \cite{akt3}. Destacamos entre las mutaciones de este gen los siguientes síntomas: discapacidad intelectual, epilepsia, autismo, malformaciones capilares en la piel, entre otros.

Como hemos observado anteriormente, existe una alta correlación en los genes mostrados en nuestro trabajo con el fenotipo, e invitamos a futuros investigadores que sigan nuestra línea de trabajo y comprueben experimentalmente que los demás genes mostrados están relacionados con el fenotipo en cuestión, e idealmente, desarrollar un tratamiento.



%% So what

Los hallazgos de nuestro estudio revisten gran importancia en la relación entre la megalencefalia y sus fenotipos. Esta comprensión más profunda puede ser fundamental para los médicos, ya que les permite diagnosticar y tratar de manera más efectiva esta enfermedad cerebral. Además, el empleo de herramientas avanzadas de análisis de redes para entender las conexiones y patrones entre la enfermedad y sus manifestaciones representa un enfoque prometedor que puede impulsar el descubrimiento en este campo.

La relevancia de nuestra investigación radica en el impacto significativo que la megalencefalia puede tener en la salud y la calidad de vida de los afectados. Un mayor entendimiento de la enfermedad y sus síntomas puede guiar a los médicos hacia tratamientos más adecuados y una atención más precisa para los pacientes. Además, la aplicación de estas herramientas avanzadas de análisis de redes no solo se limita a esta enfermedad en particular, sino que puede extenderse a otros ámbitos de la medicina, potencialmente conduciendo a avances importantes en el cuidado y tratamiento de diversas enfermedades.