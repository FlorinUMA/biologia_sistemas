\section{Conclusiones}

Este estudio sobre la megalencefalia se ha desarrollado una metodología gracias a la cual se han detectado nuevos genes que ha permitido identificar y comprender mejor las limitaciones y desafíos en el enfoque de estudio acerca de la enfermedad cerebral. Los resultados obtenidos a través del enriquecimiento funcional y análisis de redes, como se ha sospechado por los nuevos avances sobre los fundamentos moleculares de este trastorno, han permitido descubrir nuevas conexiones y patrones entre la enfermedad y sus síntomas, lo que puede ser de gran ayuda para los médicos en el diagnóstico y tratamiento de la megalencefalia. Además, el uso de herramientas avanzadas de análisis de redes puede ser aplicado en otros ámbitos de la medicina, lo que puede conducir a avances importantes en el cuidado y tratamiento de diversas enfermedades. Así, este estudio representa un importante avance en la comprensión de la megalencefalia y sus implicaciones clínicas, ya que se ha descubierto que parte de los genes propuestos en la red son genes ya conocidos y relacionados con la enfermedad de manera experimental. 