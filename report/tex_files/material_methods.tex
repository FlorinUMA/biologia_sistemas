\section{Materiales y métodos}

\subsection{Descripción general}

A lo largo de la investigación, se ha hecho uso de diversos programas, herramientas y bases de datos para la adecuada puesta en marcha del estudio. Entre las bases de datos podemos encontrar \textbf{HPO (Human Phenotype Ontology)} que es una ontología formal de fenotipos humanos (https://hpo.jax.org/app/) y \textbf{StringDB} (https://string-db.org/), base de datos la cual se ha usado poder realizar la interacción entre las proteínas del fenotipo de interés. Posteriormente, se han procesado los datos obtenidos para hacer uso de ellos en R y Python (https://www.python.org/) con el paquete de \textbf{iGraph} (https://igraph.org/) y así, esbozar los grafos que se han considerado de interés en nuestro estudio.

\subsection{Herramientas y materiales usados}
% Todo este párrafo se puede resumir en dos lineas.
% En esta sección, se detallan los materiales y equipos utilizados durante el curso de la investigación. La selección precisa de materiales y la utilización de equipos adecuados son esenciales para garantizar la validez y la reproducibilidad de los resultados obtenidos. A continuación, se presenta una enumeración detallada de los materiales, incluyendo marcas, modelos y proveedores, así como la especificación de cualquier equipo o instrumento utilizado en los experimentos. \\

Los elementos que han sido utilizados durante la realización de la investigación, los materiales, incluyendo marcas, modelos y proveedores, así como la especificación de cualquier otro instrumento utilizado en los experimentos son:

Equipo Usado:
\begin{itemize}
	\item Portátil ROG STRIX G15:
	\begin{itemize}
		\item Ubuntu 22.04.3 LTS % Por dios, no pongáis windows. Bash no se puede usar en Windows CLAU: PERDON JAJAJAJAJAJAJA 
		\item Procesador: Intel(R) Core(TM) i7-10750H CPU @ 2.60 GHz
		\item RAM: 16,0 GB
		\item Sistema: Sistema operativo de 64 bits, procesador basado en x86-64
		\item Fabricante: ASUSTeK COMPUTER INC.
	\end{itemize}
\end{itemize}

Bases de datos:
\begin{itemize}
	\item \textbf{StringDB}: es una base de datos utilizada en biología de sistemas y genómica funcional. Su función principal ha sido proporcionar información sobre las interacciones entre proteínas del fenotipo de interés. La base de datos incluye datos experimentales y predicciones computacionales sobre estas interacciones, permitiendo así explorar y comprender las relaciones entre proteínas.
	\item \textbf{HPO}: Ontología de Fenotipos Humanos, es un sistema estandarizado para describir fenotipos humanos en el contexto de enfermedades genéticas. Organiza términos en una estructura jerárquica, establece relaciones entre ellos y se utiliza ampliamente en genómica clínica para analizar la asociación entre genotipos y fenotipos. Se ha usado para obtener los datos del fenotipo de Megalencefalia.
\end{itemize}

%TODO Añadir las librerías usadas en el script de R

Software:
\begin{itemize}
	\item \textbf{Visual Studio}: Es un conjunto de herramientas de desarrollo integrado (IDE) desarrollado por Microsoft. Visual Studio incluye características como resaltado de sintaxis, depuración visual, gestión de proyectos, herramientas de diseño de interfaz de usuario y control de versiones. Es ampliamente utilizado en el desarrollo de software para aplicaciones de escritorio, web y móviles, ofreciendo una plataforma integral para el ciclo de vida completo del desarrollo de software.
	\item \textbf{Pandas (https://pandas.pydata.org/)}: Pandas es una biblioteca de Python utilizada para manipular y analizar datos de manera eficiente. Por otro lado, requests es una biblioteca de Python que permite realizar solicitudes HTTP. En Jupyter, requests se utiliza para interactuar con APIs web, como la API de StringDB.
	\item \textbf{Requests (https://pypi.org/project/requests/)}: biblioteca de Python que permite realizar solicitudes HTTP. En nuestros scripts, requests se utiliza para interactuar con APIs web, como la API de StringDB.
	\item \textbf{RStudio} (https://es.wikipedia.org/wiki/RStudio): Es un entorno de desarrollo integrado (IDE) para el lenguaje de programación R. Proporciona herramientas avanzadas y una interfaz amigable, siendo ampliamente utilizado en estadísticas y análisis de datos.
	\item \textbf{iGraph}: Es una biblioteca especializada en el análisis de redes y grafos. Proporciona funciones y herramientas para crear, visualizar y analizar estructuras de red en diversos contextos, siendo especialmente útil en la representación y exploración de relaciones entre nodos en un grafo.
\end{itemize}


% Este párrafo creo que sobra fijo
% Esta enumeración exhaustiva proporciona información detallada sobre los componentes clave utilizados en la investigación, asegurando que otros investigadores puedan replicar los experimentos de manera precisa. La especificación de marcas, modelos y proveedores es fundamental para la reproducibilidad y la comprensión completa de los métodos empleados.

% ¿Debería ir en la subsección de diseño e implementación?
Para asegurar la reproducibilidad, se aconseja tener en cuenta que la versión de R utilizada es ... y la versión de Python que se empleará es ... No obstante, las librerías que se emplearán deberán estar actualizadas a versiones concreta, ya que nuevas actualizaciones podrán inutilizar el código. Para mitigar riesgos de fallo, un script automático va a ser el encargado de instalar la versión correcta en el entorno del usuario.

\subsection{Diseño y procedimiento experimental}
El presente estudio se basó en la recopilación de datos relacionados con megalencefalia desde la HPO. En concreto, se obtuvo el archivo de asociaciones de genes correspondiente al fenotipo HP:0001355 (megalencefalia). Una vez se obtuvo el archivo, se procesaron los datos en Jupyter mediante el uso de \textbf{pandas} con la cual se realizó operaciones de limpieza y transformación para garantizar la calidad y consistencia de los datos. 

Posteriormente, se empleó la API de StringDB y la biblioteca \textbf{requests} para generar un modelo de interacción entre proteínas asociadas a megalencefalia. La información se devolvió en formato JSON y se elegió la información más relevante para nuestro estudio. Una vez hecho esto, se guardaron los resultados en un fichero CSV y se prosiguió con el siguiente paso, que era generar los grafos detallados de las relaciones entre las proteínas en Rstudio. \\

Como siguiente paso, ya en Rstudio se cargaron los datos desde el archivo CSV y se creó el grafo de interacciones de la siguiente manera:

\begin{itemize}
	\item Se identificaron nodos únicos a partir de las interacciones y con ello se creó un grafo no dirigido que representa las interacciones entre proteínas.
	\item Se utilizó el algoritmo Louvain y edge betweenness para detectar comunidades de proteínas en el grafo,el primero detectando comunidades por nodos y el segundo por enlaces, a las cuales se le asignaron un color para para hacer su visualización más cómoda. El algoritmo de Louvain fue elegido para la detección de comunidades en el grafo de interacciones de proteínas debido a su eficiencia computacional y capacidad para optimizar la modularidad del grafo. Este algoritmo es reconocido por su rapidez y eficacia en el manejo de conjuntos de datos grandes, lo que resulta fundamental para aplicaciones prácticas en el análisis de interacciones biológicas. La flexibilidad del algoritmo, su capacidad para detectar comunidades superpuestas y su relativa facilidad de implementación contribuyeron a su elección como una herramienta efectiva para la identificación de patrones de agrupamiento en el contexto de comunidades de proteínas. El otro algoritmo se usará para realizar una comparación entre los 2.
	\item Se desarrolló una función para realizar análisis de enriquecimiento funcional en cada comunidad y se aplicó la función a cada comunidad identificada, proporcionando información sobre las funciones biológicas asociadas.
	\item Se aplicó el algoritmo de detección de comunidades por enlace al grafo y, de nuevo, se asignaron colores a las comunidades por enlace y se visualizó el grafo.
	\item 
	\item Se aplicó la función de enriquecimiento funcional a las comunidades identificadas por enlace y se obtuvo información sobre las funciones biológicas asociadas a estas comunidades.
	
\end{itemize}
Todos estos pasos proporcionan una visión general del flujo de trabajo, que abarcan desde la adquisición de datos en HPO hasta el análisis de enriquecimiento funcional en R, demuestran un enfoque integrado utilizando StringDB, Jupyter y R para explorar y comprender las complejas interacciones de proteínas asociadas a megalencefalia.


\subsection{Limitaciones}
Una limitación significativa de este proyecto reside en la falta de sujetos humanos reales para estudiar la megalencefalia. Dada la naturaleza de nuestro enfoque, nos basamos en datos y simulaciones, sin contar con la participación directa de individuos afectados. Aunque se ha empleado un enfoque riguroso utilizando modelos y datos disponibles, la ausencia de casos clínicos reales podría afectar la generalización de los resultados a situaciones específicas de pacientes con megalencefalia. Se recomienda cautela al extrapolar los hallazgos a poblaciones humanas y se sugiere la validación adicional mediante estudios clínicos con participantes reales en futuras investigaciones. 

