\section{Introducción}

La megalencefalia se puede traducir en un trastorno del desarrollo caracterizado por un crecimiento excesivo de las estructuras cerebrales y un aumento del número de neuronas y células gliales como consecuencia de acontecimientos anormales posnatales. Se presenta como dos desviaciones estándar del perímetro cefálico por encima de la media correspondiente a la edad del paciente. Este trastorno suele provocar epilepsia, discapacidades en el desarrollo y problemas de conducta \cite{pavone_clinical_2017}.
No se debe confundir la megalencefalia con la macrocefalia, pues bien pueden coexistir, presentan diferentes evaluaciones clínicas, pronóstico y tratamiento. 
A día de hoy, se conocen las bases moleculares de muchos de estos trastornos, lo cual permite acercarse un poco más a entender como la desregulación de ciertas vías puede conducir a la enfermedad, y con ello, un posible diagnóstico e intervención terapéutica. \cite{winden_megalencephaly_2015} \\

Cuando se habla de las consecuencias de la patología, pueden llegar a surgir enfermedades como la de Canavan \cite{avellaneda_errores_2014} y la enfermedad de Alexander \cite{hagemann_alexander_2022}, que conduce a problemas neurológicos \cite{winden_megalencephaly_2015}, y tienen en común la acumulación de sustancias tóxicas en el cerebro.

Eventualmente, cuando se habla de los familiares de una persona que padece la patología, la calidad de vida puede ser extremadamente desafiante. El cuidado de un ser querido con necesidades médicas y neurológicas complejas puede ser emocionalmente agotador y financiera y físicamente demandante \cite{olivares_jimenez_ninos_2021}. El apoyo continuo y los recursos médicos son esenciales para brindar la mejor atención posible a quienes padecen estas enfermedades y para mejorar la calidad de vida de sus familias. \\


La megalencefalia dispone de una prevalencia del 2\% de la población infantil, lo que equivale a que 2 de cada 50 infantes presentan este fenotipo \cite{sandler_neurodevelopmental_1997}. No obstante, aunque esta patología está relacionada con severas mutaciones génicas y moleculares \cite{pavone_clinical_2017}, está también ligado al autismo, en el que aproximadamente el 15\% de los niños presentan esta malformación \cite{libero_persistence_2016}. Además, también se ha observado que afecta más a varones que a mujeres \cite{noauthor_megalencephaly_nodate}. \\

Este trastorno puede ser debido a tres causas principales: a un problema metabólico, del desarrollo o lesiones cerebrales. Por lo general, las megalencefalias metabólicas están causadas por anomalías genéticas en el funcionamiento celular. En cambio, se han constatado recientemente \cite{winden_megalencephaly_2015} que las megalencefalias de origen del desarrollo están ocasionadas por desajustes en las vías de comunicación que supervisan la multiplicación, el desarrollo y el traslado de las neuronas.

En cuanto a los factores de riesgo podemos caracterizar 4 factores principalmente: Antecedentes familiares(La presencia de antecedentes familiares aumenta el riego), Trastornos genéticos (Trastornos genéticos que estén relacionados con la megalencefalia aumentan la probabilidad), Exposición prenatal a agentes teratogénicos (La exposición a infecciones virales o drogas aumenta el riesgo en el feto) y lesiones Cerebrales previas \cite{winden_megalencephaly_2015}. \\

La principal motivación de la investigación es allanar el camino hacia la comprensión y eventual tratamiento de la megalencefalia, abordando esta condición desde múltiples perspectivas. De esta manera, se posibilitará a las personas afectadas con esta anomalía cerebral el acceso a diagnósticos más precisos y tratamientos efectivos, infundiendo así esperanzada y mejorando la calidad de sus vidas de manera significativa.\\


En cuanto a trabajos relacionados, centrándose más en el ámbito biológico del gen, se puede encontrar algunos como los siguientes, que estudian más la megalencefalia a fondo. Se tiene el primer estudio que relata acerca de una variación de la megalencefalia, como es la megalencefalia capilar \cite{mirzaa_megalencephaly-capillary_2012} o \cite{pirozzi_microcephaly_2018}, que se relata información más anatómica comparándolo con la microencefalia.
