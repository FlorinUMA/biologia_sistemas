\section{Introducción}

La megalencefalia dispone de una prevalencia del 2\% de la población infantil, lo que equivale a que 2 de cada 50 infantes presentan este fenotipo \cite{sandler_neurodevelopmental_1997}. No obstante, aunque esta patología está relacionada con severas mutaciones génicas y moleculares \cite{pavone_clinical_2017}, está también ligado al autismo, en el que aproximadamente el 15\% de los niños presentan esta malformación \cite{libero_persistence_2016}. Además, también se ha observado que afecta más a varones que a mujeres \cite{noauthor_megalencephaly_nodate}. \\