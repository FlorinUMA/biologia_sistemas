\section{Introducción}

La megalencefalia se puede traducir en un trastorno del desarrollo caracterizado por un crecimiento excesivo de las estructuras cerebrales y un aumento del número de neuronas y células gliales como consecuencia de acontecimientos anormales posnatales. Se presenta como dos desviaciones estándar del perímetro cefálico por encima de la media correspondiente a la edad del paciente. Este trastorno suele provocar epilepsia, discapacidades en el desarrollo y problemas de conducta \cite{pavone_clinical_2017}.
No se debe confundir la megalencefalia con la macrocefalia, pues bien pueden coexistir, presentan diferentes evaluaciones clínicas, pronóstico y tratamiento. 
A día de hoy, se conocen las bases moleculares de muchos de estos trastornos, lo cual permite acercarse un poco más a entender como la desregulación de ciertas vías puede conducir a la enfermedad, y con ello, un posible diagnóstico e intervención terapéutica. \cite{winden_megalencephaly_2015} \\


La megalencefalia dispone de una prevalencia del 2\% de la población infantil, lo que equivale a que 2 de cada 50 infantes presentan este fenotipo \cite{sandler_neurodevelopmental_1997}. No obstante, aunque esta patología está relacionada con severas mutaciones génicas y moleculares \cite{pavone_clinical_2017}, está también ligado al autismo, en el que aproximadamente el 15\% de los niños presentan esta malformación \cite{libero_persistence_2016}. Además, también se ha observado que afecta más a varones que a mujeres \cite{noauthor_megalencephaly_nodate}. \\