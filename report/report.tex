\documentclass{bmcart}

%%%%%%%%%%%%%%%%%%%%%%%%%%%%%%%%%%%%%%%%%%%%%%
%%                                          %%
%% CARGA DE PAQUETES DE LATEX               %%
%%                                          %%
%%%%%%%%%%%%%%%%%%%%%%%%%%%%%%%%%%%%%%%%%%%%%%

%%% Load packages
\usepackage{amsthm,amsmath}
\usepackage{graphicx}
%\RequirePackage[numbers]{natbib}
%\RequirePackage{hyperref}
\usepackage[utf8]{inputenc} %unicode support
\usepackage{url}
\usepackage{array}
\usepackage{booktabs}
\usepackage[utf8]{inputenc}
%\usepackage[applemac]{inputenc} %applemac support if unicode package fails
%\usepackage[latin1]{inputenc} %UNIX support if unicode package fails


%%%%%%%%%%%%%%%%%%%%%%%%%%%%%%%%%%%%%%%%%%%%%%
%%                                          %%
%% COMIENZO DEL DOCUMENTO                   %%
%%                                          %%
%%%%%%%%%%%%%%%%%%%%%%%%%%%%%%%%%%%%%%%%%%%%%%

\begin{document}

	\begin{frontmatter}
	
		\begin{fmbox}
			\dochead{Research}
			
			%%%%%%%%%%%%%%%%%%%%%%%%%%%%%%%%%%%%%%%%%%%%%%
			%% INTRODUCIR TITULO PROYECTO               %%
			%%%%%%%%%%%%%%%%%%%%%%%%%%%%%%%%%%%%%%%%%%%%%%
			
			\title{Explorando los Misterios de la Megalencefalia: Avanzando en la Comprensión y Tratamiento de esta Anomalía Cerebral}
			
			%%%%%%%%%%%%%%%%%%%%%%%%%%%%%%%%%%%%%%%%%%%%%%
			%% AUTORES. METER UNA ENTRADA AUTHOR        %%
			%% POR PERSONA                              %%
			%%%%%%%%%%%%%%%%%%%%%%%%%%%%%%%%%%%%%%%%%%%%%%
			
			\author[
			  addressref={aff1},                   % ESTA LINEA SE COPIA IGUAL PARA CADA AUTOR
			  corref={aff1},                       % ESTA LINEA SOLO DEBE TENERLA EL COORDINADOR DEL GRUPO
			  email={mruizvillarrazo@gmail.com}   % VUESTRO CORREO ACTIVO
			]{\inits{M.A.R.V}\fnm{Miguel Ángel} \snm{Ruiz Villarrazo}} % inits: INICIALES DE AUTOR, fnm: NOMBRE DE AUTOR, snm: APELLIDOS DE AUTOR
			\author[
			  addressref={aff1},
			  email={florinb@uma.es}
			]{\inits{F.B.V}\fnm{Florín} \snm{Babusca Voicu}}
			\author[
			  addressref={aff1},
			  email={raulobreroberlanga@uma.es}
			]{\inits{R.O.B}\fnm{Raúl} \snm{Obrero Berlanga}}
			\author[
			  addressref={aff1},
			  email={claudia.vegarodriguez@uma.es}
			]{\inits{C.V.R}\fnm{Claudia} \snm{Vega Rodriguez}}
			
			%%%%%%%%%%%%%%%%%%%%%%%%%%%%%%%%%%%%%%%%%%%%%%
			%% AFILIACION. NO TOCAR                     %%
			%%%%%%%%%%%%%%%%%%%%%%%%%%%%%%%%%%%%%%%%%%%%%%
			
			\address[id=aff1]{%                           % unique id
			  \orgdiv{ETSI Informática},             % department, if any
			  \orgname{Universidad de Málaga},          % university, etc
			  \city{Málaga},                              % city
			  \cny{España}                                    % country
			}
		
		\end{fmbox}% comment this for two column layout
		
		\begin{abstractbox}
		
			\begin{abstract} % abstract
			
			%%%%%%%%%%%%%%%%%%%%%%%%%%%%%%%%%%%%%%%%%%%%%%%
			%% RESUMEN BREVE DE NO MAS DE 100 PALABRAS   %%
			%%%%%%%%%%%%%%%%%%%%%%%%%%%%%%%%%%%%%%%%%%%%%%%

            La megalencefalia es una alteración del desarrollo corporal que afecta al cráneo, desembocando en un crecimiento excesivo del encéfalo.  En la actualidad, se tienen conocimientos sobre los fundamentos moleculares y genéticos de muchos de estos trastornos, y gracias a la bioinformática se puede acelerar la comprensión de los posibles impactos de mutaciones. Por ende, implementamos un flujo de trabajo que recopila todos los genes conocidos asociados al fenotipo, para posteriormente enriquecer la información obtenida a partir otras fuentes de información, detectar posibles comunidades y generar redes de interacción génicas, y así hemos predicho nuevos genes potencialmente relacionados con la enfermedad.
			
			\end{abstract}
			
			%%%%%%%%%%%%%%%%%%%%%%%%%%%%%%%%%%%%%%%%%%%%%%
			%% PALABRAS CLAVE DEL PROYECTO              %%
			%%%%%%%%%%%%%%%%%%%%%%%%%%%%%%%%%%%%%%%%%%%%%%
			% ¡¡¡DEBEN APARECER EN EL ABSTRACT!!! %
			\begin{keyword}
			    \kwd{megalencefalia}
			    \kwd{redes de interacción génicas}
			    \kwd{detección de comunidades}
                \kwd{bioinformática}
                \kwd{genética}
			\end{keyword}
		
		
		\end{abstractbox}
	
	\end{frontmatter}
	
	
	%%%%%%%%%%%%%%%%%%%%%%%%%%%%%%%%%
	%% COMIENZO DEL DOCUMENTO REAL %%
	%%%%%%%%%%%%%%%%%%%%%%%%%%%%%%%%%
	
	\section{Introducción}

La megalencefalia dispone de una prevalencia del 2\% de la población infantil, lo que equivale a que 2 de cada 50 infantes presentan este fenotipo \cite{sandler_neurodevelopmental_1997}. No obstante, aunque esta patología está relacionada con severas mutaciones génicas y moleculares \cite{pavone_clinical_2017}, está también ligado al autismo, en el que aproximadamente el 15\% de los niños presentan esta malformación \cite{libero_persistence_2016}. Además, también se ha observado que afecta más a varones que a mujeres \cite{noauthor_megalencephaly_nodate}. \\
	\section{Materiales y métodos}

\subsection{Descripción general}

A lo largo de la investigación, se ha hecho uso de diversos programas, herramientas y bases de datos para la adecuada puesta en marcha del estudio. Entre las bases de datos podemos encontrar \textbf{HPO (Human Phenotype Ontology)} que es una ontología formal de fenotipos humanos (https://hpo.jax.org/app/) y \textbf{StringDB} (https://string-db.org/), base de datos la cual se ha usado poder realizar la interacción entre las proteínas del fenotipo de interés. Posteriormente, se han procesado los datos obtenidos para hacer uso de ellos en R y Python (https://www.python.org/) con el paquete de \textbf{iGraph} (https://igraph.org/) y así, esbozar los grafos que se han considerado de interés en nuestro estudio.

\subsection{Herramientas y materiales usados}
% Todo este párrafo se puede resumir en dos lineas.
% En esta sección, se detallan los materiales y equipos utilizados durante el curso de la investigación. La selección precisa de materiales y la utilización de equipos adecuados son esenciales para garantizar la validez y la reproducibilidad de los resultados obtenidos. A continuación, se presenta una enumeración detallada de los materiales, incluyendo marcas, modelos y proveedores, así como la especificación de cualquier equipo o instrumento utilizado en los experimentos. \\

Los elementos que han sido utilizados durante la realización de la investigación, los materiales, incluyendo marcas, modelos y proveedores, así como la especificación de cualquier otro instrumento utilizado en los experimentos son:

Equipo Usado:
\begin{itemize}
	\item Portátil ROG STRIX G15:
	\begin{itemize}
		\item Ubuntu 22.04.3 LTS % Por dios, no pongáis windows. Bash no se puede usar en Windows CLAU: PERDON JAJAJAJAJAJAJA 
		\item Procesador: Intel(R) Core(TM) i7-10750H CPU @ 2.60 GHz
		\item RAM: 16,0 GB
		\item Sistema: Sistema operativo de 64 bits, procesador basado en x86-64
		\item Fabricante: ASUSTeK COMPUTER INC.
	\end{itemize}
\end{itemize}

Bases de datos:
\begin{itemize}
	\item \textbf{StringDB}: es una base de datos utilizada en biología de sistemas y genómica funcional. Su función principal ha sido proporcionar información sobre las interacciones entre proteínas del fenotipo de interés. La base de datos incluye datos experimentales y predicciones computacionales sobre estas interacciones, permitiendo así explorar y comprender las relaciones entre proteínas.
	\item \textbf{HPO}: Ontología de Fenotipos Humanos, es un sistema estandarizado para describir fenotipos humanos en el contexto de enfermedades genéticas. Organiza términos en una estructura jerárquica, establece relaciones entre ellos y se utiliza ampliamente en genómica clínica para analizar la asociación entre genotipos y fenotipos. Se ha usado para obtener los datos del fenotipo de Megalencefalia.
\end{itemize}

%TODO Añadir las librerías usadas en el script de R

El conjunto de software utilizado en el proyecto incluye varias herramientas clave. En primer lugar, se emplea \textbf{Visual Studio Code (VS Code)} \cite{vscode}, un editor de código fuente gratuito y de código abierto desarrollado por Microsoft. Este editor es altamente personalizable y extensible, con un rico ecosistema de extensiones que permiten ampliar su funcionalidad. Adicionalmente, VS Code ofrece soporte para una amplia gama de lenguajes de programación, siendo Python uno de los principales enfoques.

Para la manipulación eficiente y el análisis de datos, se utiliza la biblioteca de Python llamada \textbf{Pandas} \cite{pandas}. Esta biblioteca es conocida por proporcionar estructuras de datos flexibles y herramientas de análisis de datos eficaces.

En el ámbito de las solicitudes HTTP y la interacción con APIs web, se integra la biblioteca \textbf{Requests} \cite{requests} de Python. Esta biblioteca facilita la realización de solicitudes HTTP, siendo esencial para la comunicación con APIs web, como la API de StringDB.

En el contexto del lenguaje de programación R, se utiliza el entorno de desarrollo integrado (IDE) \textbf{RStudio}. Rstudio \cite{rstudio} ofrece herramientas avanzadas y una interfaz amigable, siendo ampliamente utilizado en el ámbito de estadísticas y análisis de datos.

Complementando estas herramientas, se incorpora la biblioteca \textbf{iGraph}, especializada en el análisis de redes y grafos. iGraph proporciona funciones y herramientas para la creación, visualización y análisis de estructuras de red en diversos contextos. Esta biblioteca resulta especialmente útil en la representación y exploración de relaciones entre nodos en un grafo. En resumen, la combinación de Visual Studio Code, Pandas, Requests, RStudio e iGraph proporciona un conjunto integral de herramientas para el desarrollo, análisis y visualización de datos en el proyecto.


% Este párrafo creo que sobra fijo
% Esta enumeración exhaustiva proporciona información detallada sobre los componentes clave utilizados en la investigación, asegurando que otros investigadores puedan replicar los experimentos de manera precisa. La especificación de marcas, modelos y proveedores es fundamental para la reproducibilidad y la comprensión completa de los métodos empleados.

% ¿Debería ir en la subsección de diseño e implementación?
Para asegurar la reproducibilidad, se aconseja tener en cuenta que la versión de R utilizada es ... y la versión de Python que se empleará es ... No obstante, las librerías que se emplearán deberán estar actualizadas a versiones concreta, ya que nuevas actualizaciones podrán inutilizar el código. Para mitigar riesgos de fallo, un script automático va a ser el encargado de instalar la versión correcta en el entorno del usuario.

\subsection{Diseño y procedimiento experimental}
El presente estudio se basó en la recopilación de datos relacionados con megalencefalia desde la HPO. En concreto, se obtuvo el archivo de asociaciones de genes correspondiente al fenotipo HP:0001355 (megalencefalia). Una vez se obtuvo el archivo, se procesaron los datos en Jupyter mediante el uso de \textbf{pandas} con la cual se realizó operaciones de limpieza y transformación para garantizar la calidad y consistencia de los datos. 

Posteriormente, se empleó la API de StringDB y la biblioteca \textbf{requests} para generar un modelo de interacción entre proteínas asociadas a megalencefalia. La información se devolvió en formato JSON y se elegió la información más relevante para nuestro estudio. Una vez hecho esto, se guardaron los resultados en un fichero CSV y se prosiguió con el siguiente paso, que era generar los grafos detallados de las relaciones entre las proteínas en Rstudio. \\

Como siguiente paso, ya en Rstudio se cargaron los datos desde el archivo CSV y se creó el grafo de interacciones de la siguiente manera:

Durante el análisis de interacciones de proteínas, se identificaron nodos únicos a partir de las interacciones, generando un grafo no dirigido que representa estas relaciones. Se utilizaron los algoritmos Louvain \cite{louvain} y edge betweenness \cite{cluster_edge_betweenness} para identificar comunidades en un grafo de interacciones de proteínas, siendo Louvain elegido por su eficiencia y capacidad para optimizar la modularidad en grandes conjuntos de datos. Este algoritmo destaca por su rapidez en el manejo de datos extensos y su flexibilidad para detectar comunidades superpuestas. Las comunidades fueron visualizadas mediante colores para mayor claridad. El segundo algoritmo, edge betweenness, se utilizará para una comparación adicional entre ambos enfoques.

Además, se desarrolló una función para realizar análisis de enriquecimiento funcional en cada comunidad y se aplicó la función a cada comunidad identificada, proporcionando información sobre las funciones biológicas asociadas. Sumado a esto, se aplicó el algoritmo de detección de comunidades por enlace al grafo y, de nuevo, se asignaron colores a las comunidades por enlace y se visualizó el grafo. Se compararon las dos comunidades a través de la modularidad (REF), y se mostrará una gráfica posteriormente donde se podrá ver la comparación. También, se ha producido una comparación de las dos comunidades, observando qué nodos se encontraban en el mismo cluster y cuáles no. Por último, se ha aplicado la función de enriquecimiento funcional a las comunidades identificadas por enlace con el método ganador y se obtuvo información sobre las funciones biológicas asociadas a estas comunidades con que conjunto de genes dentro de la comunidad lo provocaban.

Todos lo descrito proporcionan una visión general del flujo de trabajo, que abarcan desde la adquisición de datos en HPO hasta el análisis de enriquecimiento funcional en R, demuestran un enfoque integrado utilizando StringDB, Jupyter y R para explorar y comprender las complejas interacciones de proteínas asociadas a megalencefalia.




	\section{Resultados}

Los resultados finales que se han obtenido son el enriquecimiento funcional de las poblaciones que fueron generadas aplicando la metodología descrita anteriormente.

En primer lugar, se obtuvo la división de población a través del algoritmo de Louvain, que, como se ha observado en la Figura \ref{fig:Grafo_louvain}, existen 5 comunidades bastante pronunciadas.

\begin{figure}[!]
	\centering
	\includegraphics[width=0.8\textwidth]{figures/grafo_colores_comunidades.png}
	\caption{Grafo louvain}
	\label{fig:Grafo_louvain}
\end{figure}

También, se ha obtenido la división por el algoritmo de edge betweenness, que como se ha observado en la Figura \ref{fig:Grafo_between}, son las mismas comunidades pero con una ligera diferencia entre dos comunidades.

\begin{figure}[!]
  \centering
  \includegraphics[width=0.8\textwidth]{figures/grafo_alternativo_comunidades.png}
  \caption{Grafo edge betweenness.}
  \label{fig:Grafo_between}
\end{figure}

Para facilitar la comprensión de los diferentes modelos, se muestra en la Tabla \ref{tabla:nodos_enlaces_diferentes} una comparación por nodos. Las filas indican los nodos, La primera columna representa el algoritmo de louvain, y la segunda a la de edge betweenness. Los valores de las celdas indican en que comunidad se encuentra este nodo. Finalmente, en la última columna se muestra si el nodo es asociado a la misma comunidad ("iguales"), o bien, si pertenecen a diferentes comunidades ("diferentes).

\newcolumntype{C}{>{\centering\arraybackslash} m{3cm} }
\begin{table}[!]
    \label{tabla:nodos_enlaces_diferentes}
 	\caption{Descripción de Nodos, Enlaces y Diferentes}
	\centering
	\begin{tabular}{|C|C|C|}
    \toprule
    Nodos & Enlaces & Diferentes \\
    \midrule
     1 & 1 & Iguales \\
     2 & 2 & Iguales \\
     3 & 3 & Iguales \\
    4 & 4 & Iguales \\
     2 & 2 & Iguales \\
     4 & 4 & Iguales \\
    2 & 2 & Iguales \\
     4 & 4 & Iguales \\
     5 & 5 & Iguales \\
     4 & 4 & Iguales \\
     1 & 1 & Iguales \\
     4 & 4 & Iguales \\
     2 & 2 & Iguales \\
     4 & 4 & Iguales \\
     2 & 2 & Iguales \\
     1 & 2 & Diferentes \\
     2 & 2 & Iguales \\
     1 & 2 & Diferentes \\
     2 & 2 & Iguales \\
     1 & 1 & Iguales \\
     3 & 3 & Iguales \\
     1 & 2 & Diferentes \\
     2 & 2 & Iguales \\
     3 & 3 & Iguales \\
    4 & 4 & Iguales \\
     4 & 4 & Iguales \\
     5 & 5 & Iguales \\
     4 & 4 & Iguales \\
 		\bottomrule
 	\end{tabular}
\end{table}

Posteriormente, se calculó la modularidad y se representó en la Figura \ref{fig:comparacion_grafos}.

\begin{figure}
  \centering
  \includegraphics[width=0.8\textwidth]{figures/comparacion_grafos.png}
  \caption{Comparación con modularidad.}
  \label{fig:comparacion_grafos}
\end{figure}

Finalmente, se obtuvo el enriquecimiento funcional del mejor algoritmo basándonos en la modularidad. Para representar las funciones asociadas a cada comunidad, se han creado tantas tablas como comunidades (ver Figuras \ref{tabla:genes_funciones_1}, \ref{tabla:genes_funciones2},  \ref{tabla:genes_funciones3}, \ref{tabla:genes_funciones4} y \ref{tabla:genes_funciones5}), en las que, en la primera columna, se muestra el conjunto de nodos dentro de la comunidad con una función en común, y en la segunda columna, la propia función.

\begin{table}[!]
    \label{tabla:genes_funciones_1}
 	\caption{Descripción de Genes y Funciones de la comunidad 1}
	\centering
	\begin{tabular}{|C|C|}
    \toprule
    \textbf{Genes} & \textbf{Funciones} \\
    \midrule
    TSC2, TSC1, MTOR, PTEN, TBC1D7, AKT3 & negative regulation of intracellular signal transduction (GO:1902532) \\
    \bottomrule
 	\end{tabular}
\end{table}

\begin{table}[!]
    \label{tabla:genes_funciones2}
 	\caption{Descripción de Genes y Funciones de la comunidad 2}
	\centering
	\begin{tabular}{|C|C|}
    \toprule
    \textbf{Genes} & \textbf{Funciones} \\
    \midrule
    PIK3R2, KRAS, PIK3CA, FGF8, FGFR3, NRAS, HRAS, FGFR2 & MAPK cascade (GO:0000165) \\
    PIK3R2, KRAS, PIK3CA, FGF8, FGFR3, NRAS, HRAS, FGFR2 & Transmembrane receptor protein tyrosine kinase signaling pathway (GO:0007169) \\
    PIK3R2, KRAS, PIK3CA, FGF8, FGFR3, NRAS, HRAS, FGFR2 & Positive regulation of intracellular signal transduction (GO:1902533) \\
    \bottomrule
 	\end{tabular}
\end{table}



\begin{table}[!]
    \label{tabla:genes_funciones3}
 	\caption{Descripción de Genes y Funciones de la comunidad 3}
	\centering
	\begin{tabular}{|C|C|}
    \toprule
    \textbf{Genes} & \textbf{Funciones} \\
    \midrule
    NPRL2, NPRL3, DEPDC5 & Cellular response to amino acid starvation (GO:0034198) \\
    NPRL2, NPRL3, DEPDC5 & Negative regulation of TOR signaling (GO:0032007) \\
    NPRL2, NPRL3, DEPDC5 & Regulation of TOR signaling (GO:0032006) \\
    NPRL2, NPRL3, DEPDC5 & Cellular response to starvation (GO:0009267) \\
    NPRL2, NPRL3, DEPDC5 & Negative regulation of intracellular signal transduction (GO:1902532) \\
    \bottomrule
 	\end{tabular}
\end{table}

\begin{table}[!]
    \label{tabla:genes_funciones4}
 	\caption{Descripción de Genes y Funciones de la comunidad 4}
	\centering
	\begin{tabular}{|C|C|}
    \toprule
    \textbf{Genes} & \textbf{Funciones} \\
    \midrule
    SMO, SIX3, DISP1, SHH, GAS1, PTCH1, GLI2, ZIC2, CDON & Smoothened signaling pathway (GO:0007224) \\
    SMO, SIX3, DISP1, SHH, GAS1, PTCH1, GLI2, ZIC2, CDON & Negative regulation of transcription, DNA-templated (GO:0045892) \\
    \bottomrule
 	\end{tabular}
\end{table}



\begin{table}[!]
    \label{tabla:genes_funciones5}
 	\caption{Descripción de Genes y Funciones de la comunidad 5}
	\centering
	\begin{tabular}{|C|C|}
    \toprule
    \textbf{Genes} & \textbf{Funciones} \\
    \midrule
    NODAL, TDGF1 & BMP signaling pathway (GO:0030509) \\
    NODAL, TDGF1 & Cellular response to BMP stimulus (GO:0071773) \\
    NODAL, TDGF1 & Transmembrane receptor protein serine/threonine kinase signaling pathway (GO:0007178) \\
    NODAL, TDGF1 & Positive regulation of protein phosphorylation (GO:0001934) \\
    NODAL, TDGF1 & Positive regulation of cell proliferation (GO:0008284) \\
    NODAL, TDGF1 & Regulation of apoptotic process (GO:0042981) \\
    NODAL, TDGF1 & Regulation of transcription from RNA polymerase II promoter (GO:0006357) \\
    \bottomrule
 	\end{tabular}
\end{table}





	\section{Discusión}

\subsection{Limitaciones}
Una limitación significativa de este proyecto reside en la falta de sujetos humanos reales para estudiar la megalencefalia. Dada la naturaleza de nuestro enfoque, nos basamos en datos y simulaciones, sin contar con la participación directa de individuos afectados. Aunque se ha empleado un enfoque riguroso utilizando modelos y datos disponibles, la ausencia de casos clínicos reales podría afectar la generalización de los resultados a situaciones específicas de pacientes con megalencefalia. Se recomienda cautela al extrapolar los hallazgos a poblaciones humanas y se sugiere la validación adicional mediante estudios clínicos con participantes reales en futuras investigaciones. 

	\section{Conclusiones}

Este estudio sobre la megalencefalia se ha desarrollado una metodología gracias a la cual se han detectado nuevos genes que ha permitido identificar y comprender mejor las limitaciones y desafíos en el enfoque de estudio acerca de la enfermedad cerebral. Los resultados obtenidos a través del enriquecimiento funcional y análisis de redes, como se ha sospechado por los nuevos avances sobre los fundamentos moleculares de este trastorno, han permitido descubrir nuevas conexiones y patrones entre la enfermedad y sus síntomas, lo que puede ser de gran ayuda para los médicos en el diagnóstico y tratamiento de la megalencefalia. Además, el uso de herramientas avanzadas de análisis de redes puede ser aplicado en otros ámbitos de la medicina, lo que puede conducir a avances importantes en el cuidado y tratamiento de diversas enfermedades. Así, este estudio representa un importante avance en la comprensión de la megalencefalia y sus implicaciones clínicas, ya que se ha descubierto que parte de los genes propuestos en la red son genes ya conocidos y relacionados con la enfermedad de manera experimental. 
	
	
	%%%%%%%%%%%%%%%%%%%%%%%%%%%%%%%%%%%%%%%%%%%%%%
	%% OTRA INFORMACIÓN                         %%
	%%%%%%%%%%%%%%%%%%%%%%%%%%%%%%%%%%%%%%%%%%%%%%
	
	\begin{backmatter}
	
		% \section*{Abreviaciones}%% No hay ninguna, creo
		
		\section*{Disponibilidad de datos y materiales}%% if any
			GitHub: \url{https://github.com/FlorinUMA/biologia_sistemas}
		
		\section*{Contribución de los autores}
			C.V.R: Responsable parcial de la creación del código de R, redactora de la metodología y parte de la discusión.
			F.B.V: Responsable de la creación del código de Python, gestión de bibliografías, pruebas y supervisión de la automatización, redacción del abstract, conclusión y revisor general del documento.
			M.A.R.V: Responsable de generar todo el flujo de trabajo en el script de bash y moldear los archivos de código al mismo. Además de la contribución en diversas partes del proyecto como introducción o discusión.
		
		
		%%%%%%%%%%%%%%%%%%%%%%%%%%%%%%%%%%%%%%%%%%%%%%%%%%%%%%%%%%%%%%%%%%%%%%%%%%%%%%%%%%%%%%%%
		%% BIBLIOGRAFIA: no teneis que tocar nada, solo sustituir el archivo bibliography.bib %%
		%% por el que hayais generado vosotros                                                %%
		%%%%%%%%%%%%%%%%%%%%%%%%%%%%%%%%%%%%%%%%%%%%%%%%%%%%%%%%%%%%%%%%%%%%%%%%%%%%%%%%%%%%%%%%
		
		\bibliographystyle{bmc-mathphys} % Style BST file (bmc-mathphys, vancouver, spbasic).
		\bibliography{bibliography/references}      % Bibliography file (usually '*.bib' )
	
	\end{backmatter}
\end{document}
